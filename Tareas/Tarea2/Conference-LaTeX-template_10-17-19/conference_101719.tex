\documentclass[conference]{IEEEtran}
\IEEEoverridecommandlockouts
% The preceding line is only needed to identify funding in the first footnote. If that is unneeded, please comment it out.
\usepackage[hidelinks]{hyperref}
\urlstyle{same}
\usepackage[spanish,es-tabla]{babel}
\usepackage{cite}
\usepackage{amsmath,amssymb,amsfonts}
\usepackage{algorithmic}
\usepackage{graphicx}
\usepackage{textcomp}
\usepackage{xcolor}
\decimalpoint
\renewcommand{\labelitemi}{$\bullet$}
\def\BibTeX{{\rm B\kern-.05em{\sc i\kern-.025em b}\kern-.08em
    T\kern-.1667em\lower.7ex\hbox{E}\kern-.125emX}}
    
    
 %-------------------------------------------------------------------------------
%                            Libreria de codigos                               %
%-------------------------------------------------------------------------------
% Paquetes necesarios
\usepackage{listings}
\usepackage{xcolor}
\usepackage{comment}

% Tipos de letra personalizadas
\def\lstbasicfont{\fontfamily{pcr}\selectfont\scriptsize}
\def\vhdlbasicfont{\fontfamily{cmtt}\selectfont\scriptsize}

% Colores personalizados
\definecolor{codegreen}{rgb}{0,0.6,0}
\definecolor{codepurple}{rgb}{0.58,0,0.82}

\definecolor{codegray}{rgb}{0.5,0.5,0.5}
\definecolor{backcolour}{rgb}{0.95,0.95,0.92}
\definecolor{codeorange}{RGB}{254, 100, 35}

% Deficion de lenguajes perzonalizados

% Definicion de lenguaje MATLAB
\lstdefinelanguage{matlabfloz}{%
  alsoletter={...},%
  morekeywords={%                             % keywords
		break,case,catch,classdef,continue,else,
		elseif,end,for,function,global,if,
		otherwise,parfor,persistent,
		return,spmd,switch,try,while,...},        % Use the matlab "iskeyword" command to get those
  comment=[l]\%,                              % comments
  morecomment=[l]...,                         % comments
  morecomment=[s]{\%\{}{\%\}},                % block comments
  morestring=[m]'                             % strings 
}[keywords,comments,strings]%

% Estilos MATLAB
\lstdefinestyle{MATLAB}{
	frame=single,
	rulecolor=\color{black},
	framexleftmargin=4mm,
	xleftmargin=2mm,
	language=matlabfloz,
  commentstyle=\color{codegreen},
  keywordstyle=\color{blue}, %magenta
  numberstyle=\tiny\color{black},
  stringstyle=\color{codepurple},
  basicstyle=\lstbasicfont\scriptsize,
  breakatwhitespace=false,         
  breaklines=true,                 
  captionpos=b,                    
  keepspaces=true,                 
  numbers=left,                    
  numbersep=5pt,                  
  showspaces=false,                
  showstringspaces=false,
  showtabs=false,                  
  tabsize=2    
}

\lstdefinestyle{PYTHON}{
	frame=single,
	rulecolor=\color{black},
	framexleftmargin=5mm,
	xleftmargin=10mm,
	language=python,
    backgroundcolor=\color{backcolour},   
    commentstyle=\color{codegreen},
    keywordstyle=\color{blue}, %magenta
    numberstyle=\tiny\color{black},
    stringstyle=\color{codeorange},
    basicstyle=\lstbasicfont\footnotesize,
    breakatwhitespace=false,         
    breaklines=true,                 
    captionpos=b,                    
    keepspaces=true,                 
    numbers=left,                    
    numbersep=5pt,                  
    showspaces=false,                
    showstringspaces=false,
    showtabs=false,                  
    tabsize=2,
    otherkeywords = {show,arange}  
}


\lstdefinestyle{CC}{
	frame=single,
	rulecolor=\color{black},
	framexleftmargin=5mm,
	xleftmargin=10mm,
	language=c++,
    backgroundcolor=\color{backcolour},   
    commentstyle=\color{codegreen},
    keywordstyle=\color{blue}, %magenta
    numberstyle=\tiny\color{black},
    stringstyle=\color{codeorange},
    basicstyle=\lstbasicfont\footnotesize,
    breakatwhitespace=false,         
    breaklines=true,                 
    captionpos=b,                    
    keepspaces=true,                 
    numbers=left,                    
    numbersep=5pt,                  
    showspaces=false,                
    showstringspaces=false,
    showtabs=false,                  
    tabsize=2,
    otherkeywords = {show,arange}  
}


\lstdefinestyle{BASH}{
	frame=single,
	rulecolor=\color{black},
	framexleftmargin=4mm,
	xleftmargin=2mm,
	language=bash,
    %backgroundcolor=\color{backcolour},   
    commentstyle=\color{codegreen},
    keywordstyle=\color{blue}, %magenta
    numberstyle=\tiny\color{black},
    stringstyle=\color{codeorange},
    basicstyle=\lstbasicfont\footnotesize,
    breakatwhitespace=false,         
    breaklines=true,                 
    captionpos=b,                    
    keepspaces=true,                 
    numbers=left,                    
    numbersep=5pt,                  
    showspaces=false,                
    showstringspaces=false,
    showtabs=false,                  
    tabsize=2,
    %otherkeywords = {show,arange}  
}

\renewcommand{\lstlistingname}{Código}% Listing -> Algorithm
\renewcommand{\lstlistlistingname}{Lista de códigos}% 
\begin{document}

\title{Algoritmo genético aplicado a función de 10 variables\\
%%{\footnotesize \textsuperscript{*}Note: Sub-titles are not captured in Xplore and should not be used}
}

\author{\IEEEauthorblockN{ Ciro Fabian Bermudez Marquez}
INAOE\\
Mexico, Puebla \\
\url{cirofabian.bermudez@gmail.com}
}

\maketitle

\begin{abstract}
En este trabajo se pone a prueba el algoritmo genético (AG) para  un problema de minimización de una función de 10 variables implementado en python.
\end{abstract}

\begin{IEEEkeywords}
AG, python, heuristica.
\end{IEEEkeywords}

\section{Introducción}

Los métodos de búsqueda heurística consisten en añadir información, basándose en el espacio estudiado hasta ese momento para hacer una búsqueda inteligente. Para problemas no lineales y multimodales se justifica usar heurísticas.

Las heurísticas se deben usar si el problema tiene tres o más variables y en este trabajo se estudiará el algoritmo genético (AG) aplicado a una función de 10 variables.

\begin{equation}
 f_{1}(x) = 0.1 \sum_{i=1} ^{10} (x_{i} -2)(x_{i} -5) + \sin(1.5 \pi x_{i})
 \end{equation} 
 teniendo en cuenta las siguientes características para el algoritmo $p_{c} = 0.7$, $p_{m} = 0.1$, $\mu =200 $, $g=50, 100, 150$, codificación de 10 bits, espacio de búsqueda $x_{i}  = [0,7]$ para todas las $i = \{1, 2, \ldots,10\}$ .


\section{El algoritmo genético}

Para entender el algoritmo genético es necesario conocer la terminología que se utiliza.
 
\textbf{Individuo:} es un vector que codifica las variables del problema.

\textbf{Población:} una matriz, un conjunto de vectores.

Las variables para el AG se codifican en cadenas binarias, de 0 o 1. El AG codifica problemas combinarlos, no continuos.

Para cambia una cadena binaria a una variable real, se puede mapear linealmente:
\begin{equation}
\frac{x - \text{min}_{x}}{\text{max}_x - \text{min}_x} = \frac{b}{2^{p} -1}
\end{equation}
donde  la precisión es igual a 

\begin{equation}
 \text{precisión} = \frac{\text{max}_x - \text{min}_x}{2^{p}}
\end{equation} 
y $p$ es el número de bits.

Para la representación binaria se utiliza la codificación Grey, ya que esta solo cambia un bit entre cada valor.

A grandes rasgos la filosofía del algoritmo genético se puede resumir en los siguientes pasos: 
\begin{itemize}
\item Utiliza un conjunto de soluciones (se le llama población).
\item Combina soluciones para generar otras.
\item La solución que sobrevive es la mejor (la más apta).
\item Usa los operadores de selección, cruza, mutación y elitismo.
\item La población guarda la \textit{inteligencia} del algoritmo.
\end{itemize}




El algoritmo requiere las siguientes variables para su funcionamiento:
\begin{itemize}
\item Tamaño de población $\mu$
\item Numero de generaciones $g$
\item Probabilidad de cruza $p_{c}$
\item Probabilidad de mutación $p_{m}$
\item La función a optimizar.
\end{itemize}



El AG realiza las siguientes operaciones:
\begin{enumerate}
\item Se inicializa aleatoriamente la población
\item Se evaluá la población
\item Para un número de generaciones:

	\begin{enumerate}
		\item Se seleccionan dos individuos
        \item Se cruzan
        \item Su mutan y se evalúan los hijos
        \item Se aplican elitismo
        \item La población de hijos sustituye a la de padres
	\end{enumerate}
	
	\item Se reporta el mejor individuo
\end{enumerate}

\subsection{Detalles de pasos de AG}

La selección, la cruza, la mutación y el elitismo son operadores genéticos. 

La selección de los individuos se realiza por medio de un \textbf{torneo binario}.

     i1 = Se escogen dos individuos y gana el mejor
     
     
     i2 = Se escogen dos individuos y gana el mejor 

   Al inicio de cada iteración se revuelven aleatoriamente
   la población (se barajean los índices ). Y se van tomando de
   dos en dos.

La \textbf{cruza de dos puntos} consisten en intercambiar partes de las cadenas binarias entre padres e hijos y se usa una probabilidad de cruza en [0,1].

En la \textbf{mutación} se escoge aleatoriamente una posición de la cadena y se invierte el bit. Se aplica con un \textit{probabilidad de mutación} que toma valores en [0,1].

Algo importante a resaltar es que si la probabilidad de cruza y mutación son iguales a 1, el algoritmo está haciendo una búsqueda aleatoria.  Es recomendado que la probabilidad de mutación este en un valor menor que 0.2.

El \textbf{elitismo} consiste en guardar la mejor solución para garantizar convergencia, el mejor individuo se mantiene en la población.

El precio de usar una heurística es que la función del problema
se tiene que ejecutar el número de generaciones multiplicada por el
número de individuos.




\section{Resultados}

Lo primero que hay que hacer es cambiar la función objetivo del archivo \textbf{evalua.py} de la siguiente manera:


\lstinputlisting[style = PYTHON, caption =  Generar número aleatorio flotante., label = cod:random]{newton3.py}

Y haciendo las modificaciones al archivo \textbf{corre.py} con las especificaciones del problema se obtuvo lo siguiente:

\begin{table}[!hbp]   
	\caption{Puntos donde hay un máximo o mínimo.}                                                                                                                
		\centering                                       
		\begin{tabular}{cc}
			\hline                                             
			\#{} Generaciones & Mínimo \\                     
			\hline 
			50 & -3.21680104689738\\                                            
			100 & -3.224391005143813\\
			150 & -3.224504148622877\\
			\hline                                             
		\end{tabular}
	\end{table}	
	
	Sabemos que el mínimo esta en con $x = 3.652887442162$ $f_{1}(x)= $ -3.224518019 y con 150 generaciones ya nos aproximamos bastante al resultado real.


\section{Conclusiones}

El algoritmo encuentra un resultado aceptado cuando utilizamos 150 generaciones, y tiene la ventaja de no necesitar las derivadas de la función objetivo en contraposición con el método de Newton, sin embargo este  tiene la desventaja de tener que elegir correctamente los parámetros del algoritmo y que dependiendo de ellos podemos tener mejores o peores resultados, con un poco de práctica y con experiencia utilizando este tipo de algoritmos su utilidad se ve inmediatamente al notar que para resolver un problema de 10 variables encuentra una solución relativamente rápido.

\begin{thebibliography}{00}
\bibitem{b1}  Dr. Luis Gerardo de la Fraga. ``Apuntes de clase'' .
\end{thebibliography}


\end{document}
